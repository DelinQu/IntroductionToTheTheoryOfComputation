\newpage
\section{附录1:Solution}
\begin{lstlisting}[language=c]
    #include<iostream>
    #include<cmath>
    #include<cstring>
    #define M 1000000
    using namespace std;
    int Qlist[M];                       //状态序列,记录的是NFA状态号     
    int transTable0[M],transTable1[M];  //读取0 / 1对应的转移矩阵
    int lft[M];                         //左子集合
    int rgt[M];                         //右子集合
    int QlistAdjust[M];                 //状态序列,记录的是DFA状态号
    bool visit[M];                      //标记为已经访问
    bool ac[M];                         //接受状态集合
    int cnt, n, q, f;
    // 将十进制p转化为二进制位对应的下标
    int indexOfBinary(int p){
        int x = 1;
        if(p == 1)
            return 0;
        int i = 0;
        while(++i)
        {
            x <<= 1;
            if(p == x) return i;
        }
        return 0;
    }
    // a:sum , b:状态号
    int travel(int a, int b){
        while(b){
            int x = b&(-b);                     // b的一个状态x = q
            if(!(a&x))                          // 如果a&x = 0,表示sum中不含x状态,用 ^ 添加
                a ^= x;             
            b ^= x;                             // 剔除b中的的x状态
        }
        return a;                               // 返回a的结果,表示b能到达的子集合
    }
    // 搜索状态图,从起始位置p开始
    void DFS(int p){
        Qlist[cnt] = p;
        int lsum = 0, rsum = 0;
        while(p){
            int x = p&(-p);                     // 选择p的一个状态,我们才用与上p的补运算可以取出最后一个为1的位
            int y = indexOfBinary(x);           // 状态x的下标位置
            lsum = travel(lsum, transTable0[y]);// 经过0到达的子集合
            rsum = travel(rsum, transTable1[y]);// 经过1到达的子集合
            p ^= x;                             // 迭代,p =p^x 剔除已经搜索的状态x,也就是最后一位
        }
        lft[cnt] = lsum;                        // 将DFA中的状态:δ(qcnt,0)保存
        rgt[cnt] = rsum;                        // 将DFA中的状态:δ(qcnt,1)保存
        cnt++;                                  // 指针移动到下一个DFA状态
        if(!visit[lsum])                        // 如果lsum还没有被访问,则要迭代访问lsum
            visit[lsum] = 1, DFS(lsum);
        if(!visit[rsum])                        // 同上
            visit[rsum] = 1, DFS(rsum);           
    }
    int main()
    {
        int t;                                  // t个测试数据
        scanf("%d", &t);                        
        while(t--)                              // 输入测试数据
        {
            scanf("%d%d%d", &n, &q, &f);        // 状态数,起始状态集合,接受状态集合的特征串对应的整数
            for(int i = 0; i < n; i++)          // 输入转移矩阵0 / 1
                scanf("%d", &transTable0[i]);
            for(int i = 0; i < n; i++)
                scanf("%d", &transTable1[i]);
            cnt = 0;
            memset(visit, 0, sizeof(visit));
            memset(ac, 0, sizeof(ac));
            visit[q] = 1;
            // 从起始状态开始搜索状态图
            DFS(q);
    
            // 遍历标记接收状态集合,sum表示当前接受状态个数
            int sum = 0;    
            for(int i = 0; i < cnt; i++)
                if(Qlist[i]&f) ac[i] = 1, sum++;    // 标记接受状态
            for(int i = 0; i < cnt; i++)
                QlistAdjust[Qlist[i]] = i;          // 将ans记录的NFA号调整到DFA上来 
            printf("%d %d %d\n", cnt, sum, 0);      // DFA的状态数,接受状态数,起始状态的编号
            for(int i = 0, j = 0; i < cnt; i++){    // 接收状态的编号
                if(ac[i]){
                    if(j)   printf(" ");
                    printf("%d", i);
                    j++;
                }
            }
            printf("\n");
            for(int i = 0; i < cnt; i++) {          // δ(qi,0)→ui
                if(i)   printf(" ");
                printf("%d", QlistAdjust[lft[i]]);
            }
            printf("\n");
            for(int i = 0; i < cnt; i++){           // δ(qi,1)→vi
                if(i) printf(" ");
                printf("%d", QlistAdjust[rgt[i]]);
            }
            printf("\n");
        }
        return 0;
    }
\end{lstlisting}