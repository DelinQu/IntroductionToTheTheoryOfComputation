\section{Lab Steps 步骤}
\subsection{分析问题}
对于NFA转DFA的问题,常用子集法求解,可以通过求闭包的方式求得DFA的状态,然后编号即可求解问题。
但是在本题中,起始状态集合和接受状态集合都是用特征串对应的整数所表示的,举个例子:状态集合的子集{q0q1q2},其特征串就是0111,而子集{q0},其特征串就是0001。将对应的特征串转换为十进制的数字
\\在本题中,我们直接使用子集法可能并不是最佳的解题方法,因为特征串用递归的方式处理往往会更加方便!

\subsection{算法思想}

\begin{enumerate}
    \item 状态集合的子集合,采用二进制(特征)串的方式,一个子集中包含该状态,对应的特征串就为1,否则为0,比如上面状态集合的子集{q0q1q2},其特征串就是0111,而子集{q0},其特征串就是0001。
    \item 采用DFS的方式搜索整个状态图。
    \item 搜索过程中,对于每一个特征串p,用p $\oplus$ $-$p取出p的最后一个状态(最低非零位),分别根据状态转换表格搜索0,1到达的集合,维护到达状态序列Qlist和下标。
    \item 标记已访问过的状态,直到整个状态图搜索结束,根据记录的序列和下标调整标号,打印结果。
\end{enumerate}
\subsubsection{对特征串的处理:}
\begin{itemize}
    \item x$\wedge $ $-$x: lowbit运算,返回最低位1表示的值
    \item lowbit x $\oplus$ x运算: 剔除最低位的1
\end{itemize}

\newpage
\subsubsection{算法伪代码表述:}
经过上述分析,算法伪代码可以表述为:
\lstinputlisting[language=c]
{./Code/mind.cpp}

